\documentclass[14pt,a4paper]{extarticle}
\usepackage{fontspec}
\usepackage{hyphenat}
\usepackage {moresize}
\usepackage[english,russian]{babel}
\usepackage{polyglossia}
\usepackage{hyperref}
\hypersetup{
    colorlinks=true,
    linkcolor=blue,
    filecolor=.,
    urlcolor=blue,
    pdftitle={Overleaf Example},
    pdfpagemode=FullScreen,
    }
\usepackage[
  top=4cm,
  bottom=2.5cm,
  left=1.5cm,
  right=1.5cm
]{geometry}
\usepackage{indentfirst}
\usepackage{graphicx}
\usepackage{float}
\usepackage{tikz}
\usepackage{adjustbox}
\usepackage{hyperref}
\usepackage{xcolor}
\usepackage{fancyhdr}
\usepackage{needspace}
\usepackage{multirow}
\usepackage{longtable}
\usepackage{enumitem}
%--------------------------------------
\setdefaultlanguage{russian}
\definecolor{blueGray}{RGB}{0, 34, 85}
\color{blueGray}
\pagestyle{fancy}
\fancyhf{}
\fancyhead[L]

\fancyfoot[R]{\thepage}
\renewcommand{\headrulewidth}{0pt}

\setmainfont{Manrope}[
  BoldFont = *-Bold,
  UprightFont = *-Medium,
]

\begin{document}
\raggedright
\thispagestyle{empty}


\vspace{-35pt}

\section*qwr
qwr

\vspace{4pt}
дата создания: <class 'datetime.date'>

\vspace*{\fill}


\clearpage
\newpage
\setcounter{page}{2}


\section*{Ukjfd6ol} Прохождение государственной гражданской службы: поступление на \\   \vspace{1em} службу, продвижение по службе, прекращение службы. \\   \vspace{1em} Порядок поступления на службу - это процесс поступления и
процедура \\   \vspace{1em} замещения должностей. \\   \vspace{1em} Поступление на гражданскую и правоохранительную службу и ее \\   \vspace{1em} прохождение гражданами России осуществляются в добровольном порядке
(по \\   \vspace{1em} контракту). Прохождение военной службы осуществляется: гражданами
России \\   \vspace{1em} - по призыву и в добровольном порядке (по контракту);
иностранными \\   \vspace{1em} гражданами - по контракту на воинских должностях, подлежащих
замещению \\   \vspace{1em} солдатами, матросами, сержантами и старшинами в Вооруженных Силах
РФ, \\   \vspace{1em} других войсках, воинских формированиях и органах. Призыв граждан
РФ \\   \vspace{1em} производится для реализации их обязанности, предусмотренной ст.
59 \\   \vspace{1em} Конституции. \\   \vspace{1em} Законом "О системе государственной службы РФ" от 27 мая 2003 г. \\   \vspace{1em} определены общие условия поступления граждан на государственную
службу \\   \vspace{1em} по контракту. По контракту вправе поступать граждане, владеющие \\   \vspace{1em} государственным языком РФ и достигшие возраста, установленного \\   \vspace{1em} федеральным законом о виде государственной службы для прохождения \\   \vspace{1em} государственной службы данного вида.Федеральным законом о виде \\   \vspace{1em} государственной службы или законом субъекта Федерации могут быть \\   \vspace{1em} установлены дополнительные требования к гражданам при поступлении
на \\   \vspace{1em} государственную службу по контракту, учитывая специфику вида \\   \vspace{1em} государственной службы. \\   \vspace{1em} В настоящее время применяются четыре организационно-правовых \\   \vspace{1em} способа замещения должностей на государственной и муниципальной
службе: \\   \vspace{1em} ЗАЧИСЛЕНИЕ, НАЗНАЧЕНИЕ, ВЫБОРЫ И КОНКУРС. \\   \vspace{1em} Заголовок2 \\   \vspace{1em} -Зачисление на службу по личной инициативе и заявлению кандидата
на \\   \vspace{1em} должность применяется для замещения внекатегорийных должностей \\   \vspace{1em} вспомогательно-обслуживающего персонала, не относящихся к \\   \vspace{1em} государственным и муниципальным должностям, но предусмотренных
штатным \\   \vspace{1em} расписанием в целях технического обеспечения деятельности \\   \vspace{1em} государственных и муниципальных органов (технические секретари, \\   \vspace{1em} лаборанты, специалисты по компьютерному обслуживанию,
архивариусы, \\   \vspace{1em} курьеры и др.), а также для замещения младших муниципальных
должностей. \\   \vspace{1em} -Назначение как способ замещения должностей на государственной и \\   \vspace{1em} муниципальной службе охватывает многие высшие, главные, ведущие,
старшие \\   \vspace{1em} и некоторые младшие государственные и муниципальные должности \\   \vspace{1em} государственной и муниципальной службы соответственно. \\   \vspace{1em} -Конкурс на замещение вакантной государственной или муниципальной \\   \vspace{1em} должности государственной или муниципальной службы обеспечивает
право \\   \vspace{1em} граждан на равный доступ соответственно к государственной и (или) \\   \vspace{1em} муниципальной службе. Конкурс проводится среди граждан, подавших \\   \vspace{1em} заявление на участие в нем. По действующему законодательству в \\   \vspace{1em} конкурсном порядке замещаются вакантные старшие, ведущие, главные
и \\   \vspace{1em} высшиегосударственные должности федеральной и государственной
службы \\   \vspace{1em} субъектов Федерации, а также высшие, главные и старшие
муниципальные \\   \vspace{1em} должности муниципальной службы. Конкурсная комиссия и
администрация \\   \vspace{1em} органа оценивают документальные, профессиональные, личностные данные
и \\   \vspace{1em} на этой объективной основе отдают предпочтение одному из претендентов
с \\   \vspace{1em} использованием различных методик оценки. \\   \vspace{1em} Через выборы замещаются государственные должности Президента РФ,
глав \\   \vspace{1em} администраций субъектов РФ (Президенты республик в составе РФ, \\   \vspace{1em} Губернаторы краев и областей), Председателя Государственной Думы
и \\   \vspace{1em} краевой (областной) думы, а также должности председателей комитетов
всех \\   \vspace{1em} государственных органов представительной (законодательной) власти.
Еще \\   \vspace{1em} более широко выборность применяется в системе местного
самоуправления \\   \vspace{1em} (выбираемые непосредственно населением должностные лица и
должностные \\   \vspace{1em} лица, избранные представительными органами). ПРОДВИЖЕНИЕ ПО
СЛУЖБЕ \\   \vspace{1em} представляет собой назначение государственного служащего на
другую \\   \vspace{1em} государственную должность с более высоким денежным содержанием,
иным \\   \vspace{1em} должностным наименованием или без такового или присвоение ему
следующего \\   \vspace{1em} квалификационного разряда при сохранении на занимаемой должности. \\   \vspace{1em} Продвижение по службе производится исключительно по признакам \\   \vspace{1em} способности и профессиональной квалификации при обязательном наличии
в \\   \vspace{1em} первом случае – вакантной государственной должности, требуемого \\   \vspace{1em} профессионального образования, специализации, опыта работы и
соблюдении \\   \vspace{1em} других квалификационных требований; во втором случае – при наличии
стажа \\   \vspace{1em} государственной службы, особых заслуг перед государством
(выполнение \\   \vspace{1em} обязанностей особой важности и сложности), успешное и
добросовестное \\   \vspace{1em} исполнение должностных обязанностей, продолжительная и
безупречная \\   \vspace{1em} служба, рекомендаций аттестационной комиссии. По результатам \\   \vspace{1em} государственного квалификационного экзамена или аттестации \\   \vspace{1em} государственным служащим присваиваются квалификационные разряды. \\   \vspace{1em} Законодательство предусматривает два рода оснований ПРЕКРАЩЕНИЯ \\   \vspace{1em} СЛУЖЕБНЫХ ПОЛНОМОЧИЙ. Одни -исходят от самого служащего, порождаются
его \\   \vspace{1em} волей и желанием:увольнение по собственному желанию, выход на
пенсию, \\   \vspace{1em} поступление на учебу, призыв на военную службу и другие
основания, \\   \vspace{1em} предусмотренные ТК РФ. \\   \vspace{1em} Помимо оснований, предусмотренных законодательством о труде, \\   \vspace{1em} увольнение государственного и муниципального служащего может быть \\   \vspace{1em} осуществлено -по инициативе государственного или муниципального
органа в \\   \vspace{1em} следующих случаях: \\   \vspace{1em} достижения им предельного возраста для замещения должности; \\   \vspace{1em} прекращения гражданства РФ; \\   \vspace{1em} несоблюдения обязанностей и ограничений, установленных для \\   \vspace{1em} государственного служащего \\   \vspace{1em} разглашения сведений, составляющих государственную или иную \\   \vspace{1em} охраняемую законом тайну; \\   \vspace{1em} возникновение иных обстоятельств, указанных в п.3 ст.21 \\   \vspace{1em} вышеназванного Федерального закона. \\   \vspace{1em} При прекращении государственной либо муниципальной службы в связи
с \\   \vspace{1em} выходом на пенсию служащий считается находящимся в отставке и
сохраняет \\   \vspace{1em} присвоенный ему квалификационный разряд. \\   \vspace{1em}


\newpage
\section*{Контакты}
Для дополнительных технических консультаций можно обращаться на горячую линию Центра противодействия киберугрозам {\color{orange}Innostage} CyberART
\\
 \hspace{26pt} {8(800) 600-88-60}

\end{document}